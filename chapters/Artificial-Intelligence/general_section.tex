\section{General}

\textbf{Artificial Intelligence\index{artificial intelligence}} \\
\acrfull{ai} is intelligence displayed by machines. 

\textbf{Agent\index{agent}} \\
An agent is anything that can be viewed as perceiving its environment through sensors and acting upon the environment through actuators. The job of AI is to design an agent program that implements the agent function – the mapping from percepts to actions. \\
\centerline{$agent = architecture + program$}
The architecture is a computing device of some sort with physical sensors and actuators. 

\textbf{Artificial General Intelligence\index{artificial general intelligence}} \\
\acrfull{agi} is a machine that is able to solve \textit{any} intellectual task that a human being can. \acrshort{agi} is the ultimate goal in the field of \acrshort{ai} research. \acrshort{agi} is also referred to as true AI\index{true AI}, full AI\index{full AI} or strong AI\index{strong AI}. 

\textbf{Fuzzy logic} \\
\textit{Empty.}
% Computerphile about fuzzy logic: https://www.youtube.com/watch?v=r804UF8Ia4c

\textbf{Neural network} \\
\textit{Empty.}

\textbf{Cluster} \\
Groups of data.

\textbf{Unsupervised learning} \\
In machine learning, unsupervised learning is the task of finding similarities in data that is unlabeled. It's up to the unsupervised learning system to find some sort of pattern in the data, sort the data which can then be grouped together in \textit{clusters}. It's up to the user to provide the learning system with desired number of clusters - which is often problematic because that is often times a question in itself. 


\textbf{Supervised learning} \\
% https://youtu.be/qDbpYUbf3e0
In supervised learning, the system has already been trained with some labeled data (called training data) giving the system an indication of how the data should be grouped. Under the hood, the system has created a function that keeps improving as even more data is being fed. Although supervised learning is a good learning system, it has some drawbacks such as \textit{overfitting} and the need for large enough amounts of training data. Supervised learning is used in \textit{neural networks}. 


\textbf{Overfitting} \\
Overfitting is a problem in supervised learning. Since the function in a supervised learning system has been perfected with a limited variety of training data (despite large amounts), the function will have difficulties labeling data which is vastly different from what it is used to. This will cause the system to either break or labeling the data at random.




% Source 1: Russel & Norvigs - Artificial Intelligence: A Modern Approach 