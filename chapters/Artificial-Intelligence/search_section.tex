\section{Search}


\textbf{Breadth-first search\index{}} \\

\textbf{Depth-first search\index{}} \\

\textbf{Backtracking search\index{}} \\

\textbf{State space\index{}} \\

\textbf{Transition model\index{}} \\
% Two definitions: 
%   Chapter 3 is specific
%   Chapter 4.3 is general

\textbf{Local search\index{}} \\

\textbf{Hill-climbing search\index{}}
% https://www.youtube.com/watch?v=oSdPmxRCWws

\textbf{Constraint satisfaction\index{}} \\
In regular state-space search, an algorithm can do only one thing: search. In CSPs there is a choice: an algorithm can search (choose a new variable assignment from several possibilities) or do a specific type of inference  called \textit{constraint propagation}: using the constraints to reduce the number of legal values for a variable,

\textbf{Constraint propagation \ Local consistency\index{}} \\

\textbf{Node consistency\index{}} \\
\textit{Empty.}

\textbf{Arc consistency\index{}} \\
\textit{Empty.}

\textbf{Path consistency\index{}} \\
% Page 210 in Modern Approach to AI book
\textit{Empty.}