\section{Basic}
\textbf{Firmware\index{firmware} \\}
Firmware is software that is semi-permanently placed in hardware. Firmware does not disappear when 
hardware is turned off (typically stored in Flash or \acrshort{rom}). 
Firmware is typically involved with very basic low-level operations, which without a device, would be
non-functional.  

\textbf{BIOS\index{BIOS}} \\
\acrfull{bios} is a software that is saved on the computer's motherboard
and is turned on whenever the computer boots up. Its primary task is to prepare the components of the machine, so that other software (like the operating system) can boot up, run and take over the control of the machine.  

\textbf{Modular\index{modular}} \\
Refers to the design of any system composed of separate components that can be connected together. The beauty of modular architecture is that you can replace or add any one component (module) without affecting the rest of the system. The opposite of a modular architecture is an integrated architecture, in which no clear divisions exist between components. The term modular can apply to both hardware and software. Modular software design, for example, refers to a design strategy in which a system is composed of relatively small and autonomous routines that fit together. 

\textbf{Standalone application\index{standalone application}} \\
A standalone application is a application that is downloaded on your local computer and is self contained. Meaning it's not dependent on another service for it to run, unlike an web application that requires a web browser to work. (i.e. you need Chrome/Safari/Firefox to run Facebook). 

\newpage
    

