\section{Networking}
    \begin{adjustwidth}{2.5em}{0cm}
    %\parskip{5cm}   % length between each paragraph
    \parindent{.}   % no indentation after each paragraph for whole file

    %``````` APPLICATION LAYER ```````

    \subsection{Application layer}
    \textbf{DHCP\index{DHCP}} \\
    \acrfull{dhcp} is a network protocol responsible for assigning IP addresses to hosts on a network. The host sends out a broadcast and gets an answer from all the DHCP-servers nearby. It's then up to the host to choose one server and then inform all the other servers which one it chose. Considering the host is IP-less the packet is broadcasted with \textit{UDP}. A DHCP server is often found integrated inside a \textit{router}. 

    
    \textbf{DNS}    \\
    \acrfull{dns} translates an \textit{IP-address} to a website. \\
    Example: For instance, YouTube's IP address is 216.58.209.142. Without DNS you would have to type in the IP-address, instead of youtube.com.  

    
    
    \textbf{SSH\index{SSH}} \\
    \acrfull{ssh} is a cryptographic network protocol used over unsecured networks. It's commonly used to perform remote login on a machine. 
    \end{adjustwidth}
    
    \begin{adjustwidth}{2.5em}{}
    %``````` TRANSPORT LAYER ```````
    \subsection{Transport layer}
    \textbf{TCP} \\
    
    \textbf{UDP} \\
    \end{adjustwidth}
    
    %``````` NETWORK LAYER ```````
    \subsection{Network layer}
    \textbf{IP address} \\
    
    \textbf{ICMP} \\
    
   \textbf{Routing table} \\
   
    
    
    %``````` LINK LAYER ```````
    \subsection{Link layer}
    \textbf{ARP} \\
    
    \textbf{MAC address} \\
    
    \textbf{PPP} \\
    





\textbf{Firewall} \\

\textbf{FTP} \\


\textbf{Tor} \\





\newpage


