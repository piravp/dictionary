\section{Software development}

\textbf{ASP.NET} \\
\textit{.} 

\textbf{Async/await}
\textit{.}

\textbf{Binary file\index{binary file}} \\
\textit{Binary means executable code that can be run directly by the machine without the need to be compiled.}

\textbf{Framework\index{framework}}
\textit{hello}

\textbf{Git\index{git}}
Git is a version control system. 

\textbf{Native language\index{native language}} \\
\textit{A native programming language is compiled to machine code. This is code that's unique to a particular operating system and can only be executed in the environment for which it was compiled. As a result, when you're dealing with a native language you have to have a different compiler for each operating system. You'll have one compiler application for Windows, another one for Mac, another one for a distinct flavor of Linux, and so on. Example of native languages: C, C++, Objective-C }

\textbf{Managed languages\index{managed language}} \\
\textit{In contrast, managed languages are compiled to an intermediate format that works across operating systems. Typically, these languages are compatible across operating systems, and include languages such as C\# and Java. In addition, in manages languages, memory is allocated dynamically at runtime which means the programmer don't need to worry about allocationg and deallocating memory, which is periodically done by the garbage collector. }


\textbf{Wrapper\index{wrapper}} \\
\textit{In the context of software engineering, a wrapper is defined as an entity that encapsulates and hides the underlying complexity of another entity by means of well-defined interfaces.}


\textbf{Wrapper application\index{wrapper application}} \\
\textit{A wrapper can be a piece of software that provides compatibility layer to another piece of software.}

\textbf{Wrapper function\index{wrapper function}} \\
\textit{A wrapper function is a function that exists just to call another function. }


\textbf{Driver} \\
Source: \url{https://www.youtube.com/watch?v=t-aRlwLI-b0}

