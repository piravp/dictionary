\section{Web}

\subsection{General}
\textbf{API\index{API}} \\
API stands for Application Programming Interface. It's a way to let one application talk to another application through a middle-man - the API. 

\textbf{RESTAPI} \\
API stands for Application Programming Interface. It's a way to let one application talk to another application through a middle-man - the API. 

\textbf{Transpile\index{transpile}} \\
In the context of web, transpiling means transforming one script into another script. A good example would be what Babel does when developing with react - it transpiles the JavaScript from ECMAScript 6 to ES5. One might be tempted to think that from for instance C\# to IL is called transpiling. This would be wrong because the two languages have very different levels of abstraction (complexity). 

\textbf{TLS} \\ 
Transport Layer Security is an extra layer between the TCP and HTTP layer, that together forms the HTTPS protocol. The TLS does an extra step of authentication before the browser starts to recieve data from the server. During this step the TLS will check if the connected server is actually the server we want to connect to. \cite{web:ssl}


\textbf{Onion web/onion browser\index{onion web}} \\
It's called the onion web because when a request is made on the Tor-network (using onion) every request that is sent out from your computer is wrapped up in several layers of encryption - kind of like an onion. So as that request travels through multiple computers, every layer of encryption slowly peeled off until it reaches its final destination. So if you requested google.com, it will finally arrive at its destination where the content of google.com, that you requested, is again wrapped up in several layers of encryption and is sent back to you. 

\textbf{Dark web\index{dark web}} \\
The dark web is the part of the web that is commonly known as the part of the web where bad things happen. However, it's not designed for that, but the fact that the dark web uses the onion web, and thus ensures anonymity, makes it attractive for felons. The dark web is a copy of the web that communicates in a different way to ensure anonymity. \cite{web:darkWeb}


\textbf{Deep web\index{deep web}} \\
Deep web is the part of the web that is not indexed and thus not searchable by popular search engines like Google. Pages that typically aren't indexed can be pages that is only meant for the user, like your personal Facebook account, which is behind a password.

%%%%%%%%%%%%%%%%%%%%%%%%%%%%%%%%%%%%%%%%%%%%%%%%%%%%%%%%%%%%%%%
\subsection{Web development}                                %%-
%%%%%%%%%%%%%%%%%%%%%%%%%%%%%%%%%%%%%%%%%%%%%%%%%%%%%%%%%%%%%%%
\textbf{Babel\index{babel}} \\
Babel is a transpiler that "translates" ES6, the newer ECMAScript standard, to ES5 (read: old JavaScript), the standard still used by most browsers. 

\textbf{ECMAScript} \\
Browsers use ECMAScript to interpret JavaScript. From the developers perspective, they are effectively the same thing. People say "they use ECMAScript when coding" because web browsers are slow to adapt to new technology. As of now, ECMAScript 5.1 (from 2011) is the current (outdated) standard. However, ECMAScript 2015 (ES6) is the emerging standard (used by for instance React.js). However since most browsers hasn't adapted to the new standard, you need a \textit{transpiler} (like Babel) to work.  

\textbf{ECMAScript 6 } \\
ECMAScript 2015, previously known as ECMAScript (ES6)



\textbf{yarn} \\
Considerer by many to be the replacement for npm. It's a package manager that is faster and handles dependencies better than npm [source].  